

\chapter{RDB Example: PostgreSQL}

PostgreSQL (or Postgres) is a powerful open-source relational database project. It is gaining a lot of popularity recently.

\section{Brief Introduction}

PostgreSQL is claimed to be the world's most advanced open-source RDB, and it has some great features. It is a object-relational database where the user can define customized data types in the form of objects. The database functions are modularized, meaning that it has a small base installation (only 40MB) as given in Table \ref{ch:db:tab:rdbrequirements}, and the user can expand its capability by installing additional modules per required. It complies very well with the latest SQL standard, with additional powerful add-ons.

Though powerful and efficient, PostgreSQL has not grown as popular as some other databases such as Oracle, Microsoft SQL server, and MySQL. One of the reasons for this is the lack of enterprise tier support. Being under PostgreSQL license (an open-source license similar to BSD and MIT), the available support is mostly from the community. With that being said, PostgreSQL is mature and has been adopted in some large enterprises. Of course the IT department of these companies need to work hard to maintain the database. In addition, great flexibility often means a steeper learning curve, making mastering PostgreSQL generally more difficult than other aforementioned databases.

However, with the populating use of microservices where the database efficiency and performance becomes absolutely critical, PostgreSQL is gaining more and more attentions. It is especially popular when using inside containers.

Some of the main features of PostgreSQL are summarized as follows.
\begin{itemize}
	\item Object-relational database.
	\item Excellent adherence to SQL standard.
	\item Excellent performance and scalability.
	\item Multi-version concurrency control.
	\item Modularization of features, i.e., light weight base installation and scalable add-ons.
	\item Disk-based database, and does not support in-memory tables. As a compensation, it has robust caching mechanisms that speed up reading and writing.
	\item Steep learning curve.
\end{itemize}

\section{Installation and Authentication}

The installation of PostgreSQL depends on the OS. As of this writing, the latest version of RHEL 9.4 is used as an example. Both installation and authentication to the database are introduced.

\subsection{Installation}

PostgreSQL supports a large range of operating systems including Linux, MacOS, Windows, and more. The source code and very detailed documentations of PostgreSQL are available on its websites and can be downloaded to a local machine freely. Due to the lite weight of the base installation, PostgreSQL can be used conveniently in containers. The associated images can be found on Docker Hub.

To install PostgreSQL directly on a host machine, simply follow the instructions on the official website, where command line tools are provided for Linux users, and installer for windows users. For example, to install and enable PostgreSQL 16 on RHEL 9.4 running on \verb|x86_64|, use
\begin{lstlisting}
$ sudo dnf module install postgresql:16/server
$ sudo postgresql-setup --initdb
$ sudo systemctl start postgresql.service
$ sudo systemctl enable postgresql.service
\end{lstlisting}

After installation, use command \verb|pg_ctl| to control the server, and \verb|psql| to log in to the server. Notice that PostgreSQL installation may come with PgAdmin, the GUI for the DBMS, which can also be used to interact with the databases.

\subsection{Regular Authentication}

The default ``root'' user of a PostgreSQL database is ``\verb|postgres|''. A database of the same name ``\verb|postgres|'' is also created automatically. A sudoer can log in to this account as follows.
\begin{lstlisting}
$ sudo -u postgres psql postgres
postgres=#
\end{lstlisting}
where \verb|postgres=#| is the prompt of the DBMS console.

After logging in to the database as user \verb|postgres|, the user can create users and databases using
\begin{lstlisting}
postgres=# CREATE ROLE <username> LOGIN PASSWORD '<password>';
postgres=# CREATE DATABASE <database name> WITH OWNER = <username>;
\end{lstlisting}
and log in to the database as the user by
\begin{lstlisting}
$ psql -h localhost -d <database name> -U <username> -p <port>
\end{lstlisting}
where \verb|<port>| is defined in the configuration file and it is \verb|5432| by default.

\subsection{Peer Authentication}

Given that the user already logged in to the OS with his username in the OS, it is possible to use peer authentication if the login is from the local host and the PostgreSQL username is the same with the OS username.

Quickly create a user in PostgreSQL with the same username as OS using
\begin{lstlisting}
$ sudo -u postgres createuser -s $USER
\end{lstlisting}
which essentially logs in as PostgreSQL and creates a user of \verb|$USER|. With the user created, create a database for the user using
\begin{lstlisting}
$ createdb <database name>
\end{lstlisting}
Then log in to PostgreSQL using peer authentication as follows
\begin{lstlisting}
$ psql -d <database name>
<username>=>
\end{lstlisting}
where \verb|<username>=>| is the prompt to a regular user

If the database name happens to be the same with \verb|$USER|, simply use
\begin{lstlisting}
$ psql
<username>=>
\end{lstlisting}
to login to the database.

After logging in to the database, use \verb|\du| and \verb|\l| to check the users and the databases in PostgreSQL, and use \verb|SHOW port;| to check the port that PostgreSQL is running on which by default should be \verb|5432|.

\subsection{Run PostgreSQL in Containers}

To use PostgreSQL in containers, either download and run PostgreSQL images from Docker Hub, or create a Dockerfile like the following that installs and configures PostgreSQL on top of an Alpine base image. An example of such Dockerfiles is given below.
\begin{lstlisting}
FROM alpine
RUN apk update
# install postgresql
RUN apk add postgresql
RUN mkdir /run/postgresql
RUN chown postgres:postgres /run/postgresql/
USER postgres
WORKDIR /var/lib/postgresql
RUN mkdir /var/lib/postgresql/data
RUN chmod 0700 /var/lib/postgresql/data
RUN initdb -D /var/lib/postgresql/data
# prepare user scripts
RUN mkdir /var/lib/postgresql/user-scripts
RUN chmod 0700 /var/lib/postgresql/user-scripts
COPY ./start.sh /var/lib/postgresql/user-scripts
COPY ./setup_db.sql /var/lib/postgresql/user-scripts
COPY ./populate_db.py /var/lib/postgresql/user-scripts
# prepare user data
RUN mkdir /var/lib/postgresql/user-data
RUN chmod 0700 /var/lib/postgresql/user-data
COPY ./google_stock_price.csv /var/lib/postgresql/user-data
#
CMD ["/bin/sh", "/var/lib/postgresql/user-scripts/start.sh"]
\end{lstlisting}

\section{PostgreSQL-Specific New Data Types}

PostgreSQL supports a large range of data types, more than other databases such as MySQL and MariaDB. In addition to the commonly seen numeric types, character types, date and time types and boolean type, the following data types are supported.
\begin{itemize}
	\item Currency (monetary) types.
	\item Geometric types, including points, line segments, boxes, paths, polygons and circles.
	\item Network address types, such as IPv4 and IPv6 addresses and MAC addresses.
	\item Array types and associated functions such as accessing, modifying and searching arrays.
	\item Composite types and associated functions.
	\item Many more.
\end{itemize}

\section{PostgreSQL User-Defined Data Types}

User-defined data types can be used to demonstrate the object-relational database aspect of PostgreSQL. It essentially means that the attribute of an element can be an object, and can have some complex and comprehensive features.

To create customized data types, use the following syntax
\begin{lstlisting}
/*Composite Types*/
CREATE TYPE name AS
( [ attribute_name data_type [ COLLATE collation ] [, ... ] ] );

/*Enumerated Types*/
CREATE TYPE name AS ENUM
( [ 'label' [, ... ] ] );

/*Range Types*/
CREATE TYPE name AS RANGE (
SUBTYPE = subtype
[ , SUBTYPE_OPCLASS = subtype_operator_class ]
[ , COLLATION = collation ]
[ , CANONICAL = canonical_function ]
[ , SUBTYPE_DIFF = subtype_diff_function ]
[ , MULTIRANGE_TYPE_NAME = multirange_type_name ]
);

/*Base Types*/
CREATE TYPE name (
INPUT = input_function,
OUTPUT = output_function
[ , RECEIVE = receive_function ]
[ , SEND = send_function ]
[ , TYPMOD_IN = type_modifier_input_function ]
[ , TYPMOD_OUT = type_modifier_output_function ]
[ , ANALYZE = analyze_function ]
[ , SUBSCRIPT = subscript_function ]
[ , INTERNALLENGTH = { internallength | VARIABLE } ]
[ , PASSEDBYVALUE ]
[ , ALIGNMENT = alignment ]
[ , STORAGE = storage ]
[ , LIKE = like_type ]
[ , CATEGORY = category ]
[ , PREFERRED = preferred ]
[ , DEFAULT = default ]
[ , ELEMENT = element ]
[ , DELIMITER = delimiter ]
[ , COLLATABLE = collatable ]
);
\end{lstlisting}
For example,
\begin{lstlisting}
CREATE TYPE sex_type AS
enum ('M', 'F');
\end{lstlisting}
which creates a new data type called \verb|sex_type|, and it can take enumerated value of either \verb|M| or \verb|F|.

\section{PostgreSQL Stored Procedural and Functions}

Many databases including Oracle SQL, Microsoft SQL Server, MySQL, MariaDB and PostgreSQL, support stored procedures and user defined functions. This feature has been there for a very long time, but it is often not introduced in a typical introductory course. Though useful, they may introduce performance and portability issues, hence need to be handled with caution. Besides, nowadays it is often regarded as a better practice to implement the logics in the application layer, not in DBMS. Nevertheless, they are briefly introduced as follows.

The following is an example to define a function using SQL.
\begin{lstlisting}
CREATE OR REPLACE FUNCTION add_int(int, int)
RETURNS int AS
'
SELECT $1+$2;
'
LANGUAGE SQL
\end{lstlisting}
where notice that the input variable types are given in the bracket (use \verb|()| if there is no input), the output following \verb|RETURNS| (use \verb|void| if there is no output), and the SQL operations in between quotations \verb|'|, which is a deliminator and can be replaced by something else, such as the following
\begin{lstlisting}
CREATE OR REPLACE FUNCTION add_int(int, int)
RETURNS int AS
$body$
SELECT $1+$2;
$body$
LANGUAGE SQL
\end{lstlisting}

Instead of using \verb|$1|, \verb|$2| to refer to a input, names can be assigned together with types as follows.
\begin{lstlisting}
CREATE OR REPLACE FUNCTION add_int(var1 int, var2 int)
RETURNS int AS
$body$
SELECT var1+var2;
$body$
LANGUAGE SQL
\end{lstlisting}

Notice that so far we have been using SQL as the programming language for the functions, as indicated by \verb|LANGUAGE SQL|. Notice that PostgreSQL also supports other languages, such as PL/pgSQL, which is a procedural programming language supported by PostgreSQL. It closely resembles Oracle's PL/SQL language. ``PL'' in these terms represents ``Procedural Language''.

The following is a list of languages supported by PostgreSQL.
\begin{itemize}
	\item Naive installation:
	\begin{itemize}
		\item PL/pgSQL
		\item SQL
		\item C
	\end{itemize}
	\item Extension:
	\begin{itemize}
		\item PL/Python
		\item PL/Perl
		\item PL/Java
		\item PL/R
		\item and more.
	\end{itemize}
\end{itemize}

SQL is sufficient to carry out simple and straight forward tasks such as adding two numbers, as shown in the earlier example. However, when comes to handling conditional statements and loops, etc., procedural language is required. When the function is complex, it is sometimes impossible or inefficient to implement it using SQL, and PL/pgSQL and other procedural languages can solve this problem. An example is given below.
\begin{lstlisting}
CREATE OR REPLACE FUNCTION increment_value(value INT, increment INT)
RETURNS INT AS $$
DECLARE
	result INT;
BEGIN
	IF increment > 0 THEN
		result := value + increment;
	ELSE
		RAISE EXCEPTION 'Increment must be positive';
	END IF;
	RETURN result;
END;
$$ LANGUAGE plpgsql;
\end{lstlisting}

With the above been said, though convenient and powerful it might be in some use cases, it is often a good practice to keep complex logic in application layer, for logic consistency and database portability.

\section{Manipulation and Query}

PostgreSQL adopts SQL for database manipulation and query. SQL has been introduced in earlier sections, hence it is not repeated here. Only selected unique features to PostgreSQL are introduced.

While PostgreSQL server is running, enter its console using \verb|psql| from the shell. One can tell PostgreSQL console by its prompt which looks something like
\begin{lstlisting}
postgres-#
\end{lstlisting}
or
\begin{lstlisting}
postgres->
\end{lstlisting}
with ``postgres'' the current selected database. It is also possible to specify user name, default database, and other configurations when connecting to PostgreSQL server. Instead of running \verb|psql| as admin, use
\begin{lstlisting}
$ psql -h <host> -p <port> -U <username> <default_database>
\end{lstlisting}

Once in PostgreSQL console, use \verb|help| to display the basic commands, including \verb|\copyright| that shows the distribution terms, \verb|\h| to check SQL commands, \verb|\?| to check psql commands, and \verb|\q| to quit PostgreSQL console, etc. Notice that both SQL and psql commands can be used in PostgreSQL console.

Some widely used psql commands are summarized in Table \ref{ch:db:tab:psqlcommands}. Most, if not all, psql commands start with a back slash ``\verb|\|''.

\begin{table}[!htb]
	\centering \caption{Widely used psql commands.}\label{ch:db:tab:psqlcommands}
	\begin{tabularx}{\textwidth}{lX}
		\hline
		Command & Description \\ \hline
		\verb|SELECT VERSION();| & Check PostgreSQL version. \\ 
		\verb|\l| & List databases. \\ 
		\verb|\c <database>| & Switch databases. \\ 
		\verb|\d| & Describe items. By default, it lists the tables in the current database, and describe each of them. \\ 
		\verb|\dt| & List tables. \\ 
		\verb|\dv| & List views. \\ 
		\verb|\dn| & List schema. \\ 
		\verb|\df| & List functions. \\ 
		\verb|\du| & List users. \\ 
		\verb|\d <table>| & Describe a table. \\ 
		\verb|\s| & Show command history. \\ 
		\verb|\h| & Show help. \\ 
		\verb|\?| & Show psql commands. \\ 
		\verb|\!cls| & Clear screen. \\ 
		\verb|\q| & Quit DBMS shell. \\
		\hline
	\end{tabularx}
\end{table}

SQL commands, such as creating a database, have been introduced earlier, hence is not repeated here. 