\chapter{A Brief Introduction to YAML} \label{ch:yaml}

YAML is widely used as configuration files and workflow description files. Under the scope of this notebook, YAML is used at least in GitHub Actions and Kubernetes. A brief introduction to YAML and its syntax is given here.

\section{Overview}

\mync{YAML} is a human-readable data serialization language widely used for writing configuration files. The name YAML was initially interpreted as ``Yet Another Markup Language'' because the motivation for the authors to develop YAML was to simplify XML. However, later on the authors pointed out that YAML is more of a data serialization language (like JSON) than a markup language. Hence, it is now more often known as the recursive acronym ``YAML Ain't Markup Language''.

\begin{mdframed}
\Boxhead{Markup VS Serialization Languages}

Markup languages focus on the marking up of various elements in a text document. In the early days, the editor of a book often needed to put marks on the manuscripts to show where each line should go, etc., before sending it to the publisher. These marks inspired markup languages, and hence the name.

Data serialization languages, on the other hand, focus on using texts to represent data structures. Data serialization languages like JSON and YAML are able to represent ``objects'' such as Python dictionaries, JavaScript objects, etc., using its textual syntax. Data serialization languages are useful as they maintain the data structures, allowing a machine to decode the objects easily. Therefore, they are widely used in configuration files and for data storage, and machine-to-machine data transfer.

It is possible that markup and serialization languages overlaps in some applications. For example, XML, a typical markup language, can also be used to serialize data.

\end{mdframed}

\section{Syntax}

Commonly used YAML file extensions include \verb|.yaml| and \verb|.yml|. When learning YAML syntax, it is helpful to compare it side-by-side with JSON, as both of them are serialization languages and can be translated from one to the other.

Like Python, YAML uses line separation and indentation as part of its syntax. This makes it reader friendly.

YAML uses \verb|#| to lead a comment. YAML is case-sensitive. Use \verb|---| in a new line to separate a single YAML file into multiple logical sectors.

\subsection{Key-Value Pair}

Key-value pair is the most basic syntax in YAML as follows.
\begin{lstlisting}
<key>: <value>
\end{lstlisting}
For example,
\begin{lstlisting}
name: myApp
port: 9000
version: 1.1
tested: true
\end{lstlisting}

YAML is able to automatically interpret the data types of different variables. In the above example, for instance, port number $9000$ is identified as an integer, while application name \verb|myApp|, a string. To enforce it to interpret values as strings, use quotation marks. Notice that \verb|true/yes/on| and their associated \verb|false/no/off| are regarded as boolean values.


\subsection{Object and Nested Object}

Use indention to indicate object trees. See the example below. Object can be nested.
\begin{lstlisting}
<object name>:
  <key1>: <value1>
  <key2>: <value2>
  <key3>:
    <key31>: <value31>
    <key32>: <value32>
\end{lstlisting}
where object \verb|object name| contains 3 key-value pairs. The third key \verb|key3| is associated with a value who is a nested object that contains another 2 key-value pairs.

\subsection{List}

In addition to object and nested object, the value can also be list. See the example below. Be careful with the indentation when items in the list are nested objects.
\begin{lstlisting}
<list1>:
  - <item1>
  - <item2>
  - <item3>
<list2>:
  - <item1 key>: <item1 value>
  - <item2 key>: <item2 value>
  - <item3 key>: <item3 value>
<list3>:
  - <item1 key1>: <item1 value1>
    <item1 key2>: <item1 value2>
    <item1 key3>: <item1 value3>
  - <item2 key1>: <item2 value1>
    <item2 key1>: <item2 value2>
    <item2 key1>: <item2 value3>
  - <item3 key1>: <item3 value1>
    <item3 key1>: <item3 value2>
    <item3 key1>: <item3 value3>
\end{lstlisting}
In the example above, the value of \verb|<list1>| is a list of 3 variables. The value of \verb|<list2>| is a list of 3 key-value pairs in the form \verb|<itemN key>: <itemN value>|. The value of \verb|<list3>| is a list of 3 nested objects, each object containing 3 key-value pairs, in the form of \verb|<itemN keyM>: <itemN valueM>|.

Items in a list in YAML do not need to have the same data type or object structure.

For a list whose items are primitive data types, such as integer, float, boolean, string, etc., or a mix of them, it is also possible to use \verb|[item1, item2, ...]|. For example, the following two expressions are equivalent.
\begin{lstlisting}
port:
  - 9000
  - 9001
  - 9002
\end{lstlisting}
versus
\begin{lstlisting}
port: [9000, 9001, 9002]
\end{lstlisting}
where the former and later expressions are known as the block style and flow style respectively.

It is possible to have a list of only one item. Examples are given below.
\begin{lstlisting}
<list1>:
  - <item1>
<list2>: [<item2>]
<list3>:
  - <item3 key>: <item3 value>
<list4>:
  - <item4 key1>: <item4 value1>
    <item4 key2>: <item4 value2>
    <item4 key3>: <item4 value2>
\end{lstlisting}
In the above example, each list \verb|<list1>| to \verb|<list4>| has 1 item. The first two lists \verb|<list1>| and \verb|<list2>| have primitive items \verb|<item1>| and \verb|<item2>| respectively. List \verb|<list3>| has one item which is a key-value pair. List \verb|<list4>| has one item which is an object that contains 3 key-value pairs.

\subsection{Multi-line String}

To save multi-line strings as the value of a key, use \texttt{|} as follows.
\begin{lstlisting}
<key>: |
  this is the first line
  this is the second line
  this is the third line
\end{lstlisting}
This allows file-like saving, i.e., saving the content of a file as it is in the YAML file. When using file-like saving with \texttt{|}, a line break is added automatically at the end of each line.

Do not confuse multi-line strings with a long single-line string that is wrapped in YAML for interpretation convenience. To express the wrap of a single string, use \verb|>| instead of \texttt{|} as follows.
\begin{lstlisting}
<key>: >
  abc
  def
\end{lstlisting}
which is equivalent with
\begin{lstlisting}
<key>: abc def
\end{lstlisting}
Notice that when \verb|>| is used, a space instead of a line breaker is added automatically at the end of each line, making the value a long single-line string instead of a multi-line string.

In the case where the space is not needed in the long single-line string, use \verb|\| at the end of each line as follows. In this way, YAML interpret it as a simple line wrap.
\begin{lstlisting}
<key>: abc\
  def
\end{lstlisting}
which is equivalent to
\begin{lstlisting}
<key>: abcdef
\end{lstlisting}

Notice that both \texttt{|} and \verb|>| will generate a line breaker in the end of the entire text. If no such line breaker is designed, use \texttt{|-} and \verb|>-| instead.

\section{Commonly Seen YAML Use Cases}

YAML has gained its popularity among configuration files, especially when containers, cloud service and CI/CD are involved. Some commonly seen services that support YAML are given below. This is only a small portion of all the services and programs that use YAML.

\begin{itemize}
	\item GitHub Actions configuration files. GitHub Actions uses YAML as the workflow configuration files.
	\item Kubernetes configuration files. Kubernetes pods, images, services, deployments, etc., have to be configured by the developer using YAML.
	\item AWS CloudFormation. AWS CloudFormation is the manuscript using which AWS can start and configure a service automatically so that the user does not need to do everything using the dashboard. It is useful when the user needs to run similar and repetitive services. The AWS CloudFormation instruction file is written in YAML.
	\item Azure pipeline, and many other CI/CD services. Many CI/CD services uses YAML for the configuration of pipelines.
\end{itemize} 