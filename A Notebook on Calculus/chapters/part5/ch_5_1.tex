\chapter{Brief Introduction to Numerical Analysis}

Sometimes it is not possible to describe a variable or a model using analytical equations alone. For example, we may need to solve $f(x)=0$ for $x$, and there is no analytical expression of $x$ due to the complexity of $f(x)$. Or, we may need to describe the behavior of a model in the form of $f(x)$, but the true model is too difficult to calibrate precisely. In both examples, analytical solution finds its limitations.

Numerical solution serves as an alternative to the analytical solution. Though numerical solution is likely to be more computationally intensive and give less insights than the analytical solution, it finds its usefulness in many engineering problems where analytical solution is not reachable. Numerical analysis studies the effectiveness and efficiency of numerical solutions for variety of problems.

Numerical analysis goes beyond calculus. However, it still makes sense to include it as part of the calculus notebook. This is because the many fundamental methods used in numerical analysis are closely related to calculus. There are numerical analysis textbooks such as the latest edition of \cite{burden1997numerical} (as of this writing it is in its $10$th edition) that list ``review of calculus'' as its first section.

\section{Motivation}

Numerical calculation is widely used in science and engineering. Domain knowledge experts in these areas are able to demonstrate why numerical analysis is absolutely critical in their applications and their reasons may differ case by case. In this section, some of the common reasons are listed as follows.



\section{Research Interest}

\subsection{Numerical Analysis Error}

\subsection{Algorithm and Its Convergence}

\section{Tools}

