\chapter*{Preface}

This notebook introduces probability and statistics, which is one of the fundamental undergraduate-level mathematics courses for science and engineering background students at a university.

In Part I of the notebook, probability theory is introduced. Probability theory studies how likely an event is to occur or not, and it offers rich models and tools to model and describe random values and stochastic events.

In Part II of the notebook, statistics is introduced. Statistics is a collection of methods to analyze and observe insights from data, verify statistics hypothesis and draw conclusions and predictions.

In Parts III and IV of the notebook, some of the most widely known and commonly used software solutions to statistics analysis and data science are introduced. Different from Parts I and II of the notebook that focus more on theory, Part III focuses more on using tools to solve practical problems.

As a bonus, in Part V, semantic web, the database framework defined under Web 3.0, is introduced. Semantic web does not necessarily contribute to solving a specific problem. However, it might become the ``playground'' and the ``data pool'' where other data science tools need to read and write  

Key references of this notebook are summarized as follows. For probability and statistics parts:
\begin{itemize}
  \item Spiegel, Murray, John Schiller, and Alu Srinivasan. \textit{Probability and statistics.} 2020.
  \item Dekking, Frederik Michel, et al., \textit{A Modern Introduction to Probability and Statistics: Understanding why and how.} Vol. 488. London: Springer, 2005.
\end{itemize}
For data science part:
\begin{itemize}
  \item Kirill Eremenko, \textit{R Programming A-Z: R For Data Science With Real Exercises}, Udemy Course.
  \item Lakshmanan, Valliappa, Martin Görner, and Ryan Gillard. Practical Machine Learning for Computer Vision. " O'Reilly Media, Inc.", 2021.
  \item Jose Portilla, \textit{Complete TensorFlow 2 and Keras Deep Learning Bootcamp}, Udemy
\end{itemize} 

Online materials such as tutorials from YouTube, Bilibili, etc., are also used in forming this notebook. ChatGPT-4 is used as a consultant in forming this notebook. 