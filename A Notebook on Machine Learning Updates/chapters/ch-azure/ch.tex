\chapter{Microsoft Azure} \label{ch:azure}

Microsoft Build 2023 was hold on May 24. In the event the hosts claimed that ``everyone is a developer'' and ``what we would have not believed to happen very soon six months ago is happening today''. The above servers as a good summary of the AI spirit as of today: with its help, anything can happen, and anything can happen soon, and everyone can be part of it.

Azure as a platform provides integrated hardware and software supports to the developers in the community. In the hardware portion, Azure enables training and executing the AI models on cut-edge high performance computers. In the software portion, variety of ideas and tools for AI development related tasks have been proposed, including bringing Bing (Microsoft Search Engine) to ChatGPT as a plug-in for up-to-date information retrieving, Windows Copilot, Copilot stack (user-defined copilots for their APPs), Azure AI Studio (user-defined AI pipeline that involves smart plug-ins), Microsoft Fabric (unified data analysis and visualization tool, with many data models supported), and many more. Some of them are introduced in this chapter.

What is the next big thing? There are a few items in the list.
\begin{itemize}
  \item Domain knowledge AI model.
  \item Advanced copilot systems and AI platforms, with elevated user experience and data security.
  \item Smart plug-ins that enhance and complete AI model capabilities.
  \item Smaller and faster AI model while achieving similar performance tiers, meantime more edge-computing friendly.
\end{itemize}

\section{Azure Support for OpenAI}

Notice that Azure support for OpenAI and Azure AI Studio indeed share a lot of resources and interfaces. Many operations introduced in this section are done under Azure AI Studio framework in practice. OpenAI is indeed a big chunk in Azure AI Studio, as many of its features are backed up by OpenAI. Azure AI also provides other tools than OpenAI support, and those will be introduced in later sections.

The most important Azure OpenAI feature is that it allows transfer learning of AI model on customer's domain data fairly easily. Azure guarantees the data security, to make sure that customer data is isolated from one another. The transfer learning is done jointly with Azure AI Studio. When creating OpenAI resource on Azure, the customer has the choice to fine-tuning the model using customer's data via Azure AI Studio.

The customer's data may come from different URIs, including Azure data warehouse or other third-party data sources such as AWS. Everything including data used inside Azure AI Studio is kept closed only to the customer. Azure AI Studio has made it very easy to import data into the workspace for training. It is only a few button clicks from Azure UI. When the model answers a question using the customer data, it will also displays the original text from where the information is retrieved.

Plug-ins, both those recommended or even already integrated into GPT-4 and those open-source tools available on GitHub, etc., can be easily enabled in the AI model that the customer fine-tuned using its own data. Some popular plug-ins may enable the following functions: multi-language translation, advanced searching (for example, Azure Cognitive Search service, which can be enabled in the AI model as a plug-in), internet connectivity for up-to-date information, etc.

\begin{shortbox}
\Boxhead{Azure Cognitive Search}

To enhance data retrieving capability, Azure provides Azure Cognitive Search service, which is essential a NoSQL vector-based database what is good at clustering by nature. This demonstrates Azure's multi-model approach for searching (AI model fine-tuned with customer's data, together with SQL/NoSQL database).
\end{shortbox}

Some promising areas of Azure support for OpenAI include smart data retrieving, text summarizing, code generation, and copilot (current focus is mainly on text/contents generation). Some promising domains include smart health care, robot-aided data analysis, and many more.

\section{Windows Copilot Solution: Microsoft 365 Copilot}

Microsoft 365 has been made much smarter than before with the help of advanced AI model and graph database techniques. It is also more flexible, as it supports plug-ins to be integrated into the copilot. AI-assisted code generation is becoming increasingly popular. Microsoft provides multiple tools to help developers with code generation. These tools are introduced as follows.

\subsection{Microsoft 365 Copilot}

Microsoft 365 copilot is the assistant tool built-in to Microsoft 365 that helps the user to quickly retrieve data globally. The copilot has access to Teams, Outlooks, OneDrive, and other Microsoft 365 software and even Power APPs. It can respond to user's requests such as check up-coming events from a calendar, find a past email, and summarizes the email contents. The copilot has an intuitive user interface: it appears as a chatbot in Teams.

Copilot supports user-defined plug-ins. A plug-in can be created from the third-party software APIs systematically following Microsoft's instructions. The plug-in support shows the extendability of copilot.

Copilot supports semantic index, vector database, etc. Think of semantic index and vector database as Microsoft's NoSQL database that can be used to organize information efficiently. In specific, semantic index is Microsoft's graph database (similar with semantic web) to manage data in Microsoft 365 and Power APPs. Copilot is able to create, insert and retrieve data from these database.

Copilot can be used as a human-in-the-loop pipeline when multiple people are working on the same project using Microsoft 365.

Microsoft Teams Share and Microsoft Mesh are proposed. Microsoft Teams Share allows multiple people working on the same project while having a Teams meeting. Microsoft Mesh allows multiple people working in a virtual environment (think of VR games).

\subsection{AI-Assisted Code Generation}

Microsoft provides at least 2 different tools for AI-assisted code generation: GPT-assisted code generation, and GitHub copilot (also known as codex). These two models are trained from different training set, therefore, may perform differently solving the same problem.

Both tools take in the ``prompt'' (instructions and descriptions) from the developer, and generate codes accordingly. GPT-assisted code is more natural language friendly, while GitHub copilot is usually more specific on the construction of the prompt.

Regarding the copyright of the codes and images generated by the AI, there is an undergoing debate. Given that the AI model is trained mostly by public data, very likely the generated codes would be regarded as public codes and can be used freely. When comes to images, things can get a bit more complicated. The current walk around to this is to state clearly that the images are generated by AI model when such images are used.

When using GitHub copilot to generate codes, it is recommended to also check the source of the codes that is also provided by copilot along with the codes.

\section{User-Defined APPs Copilot Solution: Copilot Stack}

Here copilot refers to the ``assistance'' that comes with the software. It can be an interactive chatbot that provides data retrieving or document editing functions. In this regard, ChatGPT by itself is a general-purpose copilot. Every automated pipeline has a copilot working in the background to sort out all the tasks need to be done one by one. Here is another example, Podcast Copilot \textit{github.com/microsoft/PodcastCopilot}, which is used in Microsoft Build 2023, is an example of a copilot generating contents that can be posted to social media.

Microsoft Azure provides platform and tools, namely Copilot Stack, for developers to develop user-defined copilots for their APPs. Many features are enabled in Copilot Stack, including fine-tuning models using existing models (such as ChatGPT 3.5 and 4), using plug-ins, and calling open-source models found on GitHub.


\subsection{VS and VS Code Copilot Chat}

Visual Studio and VS Code Copilot Chat is essentially natural-language-interface GitHub Copilot built-in to VS Code that helps to make programming easier. It can create code using the comments left in the programming script, and interpret raised error, and fix the code.

\subsection{Create Plug-in for AI Models}

It is easy to create a plug-in on GitHub using its built-in tools, and call that API from AI models such as ChatGPT. Just pass the URL of the plug-in to ChatGPT, and ChatGPT will try to retrieve information using that plug-in automatically. The similar applies to the AI mode created in users own copilots built inside Copilot Stack.

The plug-in can be made either private or open to public, up to the developer's choice. The plug-in development can be put into CI/CD using GitHub actions, and deployed in containers on Azure managed by k8s.

Try this function using the demo given by \textit{github.com/Azure-Samples/openai-plugin-fastapi}.

\section{Azure AI Studio}

Azure tries its best to help the customer develop his own AI model as easily, safely and less toxic as possible. Many tools such as Promptflow has been developed.

\section{Microsoft Fabric}

